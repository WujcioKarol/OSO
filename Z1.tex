\documentclass{article}

\usepackage[MeX]{polski}
\usepackage{graphicx}
\usepackage[T1]{fontenc}
\usepackage{minted}
\usepackage{xcolor}
\usepackage{geometry}
\geometry{a4paper, total={170mm,257mm}, left=20mm, top=20mm}
\usemintedstyle{monokai}
\definecolor{bg}{HTML}{282828}

\title{Laboratorium 1}
\author{Karol Słomczynski 272223}
\date{19.03.2024}

\setminted{
    linenos=true,
    breaklines=true,
    bgcolor=bg, 
}

\begin{document}

\maketitle
\tableofcontents
\newpage
\section{Wprowadzenie}
Celem labolatorium jest przeprowadzenie analizy logów dla serwisu ssh oraz serwisu SMTP.\@ Użyłem podstawowych komend do obsługi plików tekstowych takich jak: grep, awk, sort, uniq, cut, head, tail, xargs
Utworzyłem skrypt w foldrze /pliki pod nazwą script.bash który zawiera wszystkie komendy które użyłem do analizy logów.
\section{Zadanie 1}
Zadanie pierwsze polegalo na przeprowadzeniu analizy logów serwisu ssh. Zadanie zawierało 2 pliki `secure' oraz `secure.1'.
\subsection{Podpunkt 1}
W celu zliczenia ilości prób logowania do serwisu ssh z niepoprawnymi danymi użyłem polecenia:
\begin{minted}{awk}
    grep "Failed" -c secure
    grep "Failed" -c secure.1 
\end{minted}
\subsubsection{Wyniki}
Wyniki jake otrzymałem to dla pliku secure:  16794, dla pliku secure.1:  50570
\subsection{Podpunkt 2}
W celu wyświetlenia listy adresów ip z których próbowano się zalogować użyłem polecenia:
\begin{minted}{awk}
    grep "Failed" secure | grep -Eo '[0-9]{1,3}\.[0-9]{1,3}\.[0-9]{1,3}\.[0-9]{1,3}' | uniq -c | sort -nr | uniq | head -n 30
    grep "Failed" secure.1 | grep -Eo '[0-9]{1,3}\.[0-9]{1,3}\.[0-9]{1,3}\.[0-9]{1,3}' | uniq -c | sort -nr | uniq | head -n 30
\end{minted}

\subsubsection{Wyniki}
\begin{minted}{awk}
        secure                      secure.1
    40 194.87.151.204           18 5.49.163.128   
    33 218.92.0.210             16 5.49.163.128
    28 2.189.59.146             14 218.92.0.193
    27 152.89.251.3             14 218.92.0.188
    15 61.177.173.52            14 218.92.0.153
    15 61.177.173.47            14 218.92.0.151
    15 61.177.173.37            12 58.242.82.7
    15 61.177.172.104           12 27.114.131.74
    14 194.87.151.204           12 218.92.0.193
    13 61.177.172.104           12 218.92.0.179
    13 194.87.151.204           12 218.92.0.151
    12 61.177.173.52            12 178.140.223.140
    12 61.177.173.47            12 173.249.14.238
    12 61.177.173.46            11 218.92.0.151
    12 61.177.173.37            11 188.92.77.235
    12 61.177.172.98            10 5.49.163.128
    12 61.177.172.90            10 218.92.0.152
    12 61.177.172.124           10 218.92.0.151
    12 61.177.172.114           9 218.92.0.172
    12 61.177.172.104           9 218.92.0.153
    12 194.87.151.204           9 218.92.0.152
    11 61.177.173.52            9 218.92.0.151
    11 61.177.173.47            8 77.37.165.158
    11 61.177.172.98            8 218.92.0.188
    11 124.79.242.86            8 218.92.0.153
    10 61.177.173.47            8 218.92.0.145
    10 61.177.173.46            8 178.140.30.185  
    10 61.177.172.114           8 102.152.6.176
    10 124.79.242.86            7 69.60.21.172
     9 62.193.68.91             7 68.183.22.90
\end{minted}
Jak możemy zauważyć w w pliku secure.1 próby ataku są bardziej rozproszone niż w pliku secure. Co pokazuje nam że osoba przeprowadzająca atak posiada botnet lub proxy lub VPN.
\subsection{Podpunkt 3}
W celu wyświetlenia listy krajow z których było najwięcej prób logowania użyłem polecenia:
\begin{minted}{awk}
    grep "Failed" secure | grep -Eo '[0-9]{1,3}\.[0-9]{1,3}\.[0-9]{1,3}\.[0-9]{1,3}' | uniq | xargs -I{} geoiplookup {} | sort | uniq -c | sort -nr | head -n 10
    grep "Failed" secure.1 | grep -Eo '[0-9]{1,3}\.[0-9]{1,3}\.[0-9]{1,3}\.[0-9]{1,3}' | uniq | xargs -I{} geoiplookup {} | sort | uniq -c | sort -nr | head -n 10
\end{minted}
\subsubsection{Wyniki}
\begin{minted}{awk}
                    secure                                   
    2788 GeoIP Country Edition: US, United States  
    2137 GeoIP Country Edition: CN, China                  
    742 GeoIP Country Edition: IN, India           
    664 GeoIP Country Edition: SG, Singapore       
    602 GeoIP Country Edition: DE, Germany         
    536 GeoIP Country Edition: JP, Japan           
    391 GeoIP Country Edition: RU, Russian Federation
    388 GeoIP Country Edition: KR, Korea           
    352 GeoIP Country Edition: GB, United Kingdom  
    347 GeoIP Country Edition: VN, Vietnam         
    
                    secure.1
    10691 GeoIP Country Edition: CN, China
    7028 GeoIP Country Edition: US, United States
    4876 GeoIP Country Edition: FR, France
    2127 GeoIP Country Edition: IN, India
    1822 GeoIP Country Edition: DE, Germany
    1721 GeoIP Country Edition: BR, Brazil
    1610 GeoIP Country Edition: KR, Korea, Republic of
    1562 GeoIP Country Edition: SG, Singapore
    1336 GeoIP Country Edition: CA, Canada
    1211 GeoIP Country Edition: GB, United Kingdom
\end{minted}
\subsection{Podpunkt 4}
W celu wyświetlenia listy użytkowników lokalnych na których było najwięcej prób ataku użyłem polecenia:
\begin{minted}{awk}
    grep 'Failed' secure | awk '{print $9,$11}' | awk '{if ($1 == "invalid") {print $2} else if (match($1, /^[0-9]{1,3}\.[0-9]{1,3}\.[0-9]{1,3}\.[0-9]{1,3}$/)) {print $1}}' | sort | uniq -c | sort -nr | head -n 20
    grep 'Failed' secure.1 | awk '{print $9,$11}' | awk '{if ($1 == "invalid") {print $2} else if (match($1, /^[0-9]{1,3}\.[0-9]{1,3}\.[0-9]{1,3}\.[0-9]{1,3}$/)) {print $1}}' | sort | uniq -c | sort -nr | head -n 20
\end{minted}
\subsubsection{Wyniki}
\begin{minted}{awk}
      secure                     secure.1
    391 admin                  1314 admin
    239 test                   1120 test
    219 ubuntu                  752 user
    157 user                    739 ubuntu
    121 wang                    613 ftpuser
     85 oracle                  492 postgres
     84 postgres                481 oracle
     79 zhang                   309 nagios
     70 ftpuser                 242 guest
     51 testuser                227 git
     39 chen                    223 support
     36 ali                     220 mysql
     34 test1                   212 teamspeak
     34 administrator           208 deploy
     33 yang                    205 hadoop
     33 steam                   197 minecraft
     32 pi                      179 testuser
     30 user1                   153 zabbix
     30 tomcat                  153 www
     30 guest                   143 teamspeak3
\end{minted}
\subsubsection{Wnioski}
Jak można zauważyć zarówno w pliku secure i secure.1 ataki są przeprowadzane na konta o nazwach domyślnych takich jak admin, test, user, ubuntu, ftpuser, postgres, oracle. Moim zdaniem atakujący liczy, że któryś z `domyślnych' użytkowników będzie miało słabe hasło.
\subsection{Podpunkt 5}
W celu wyświelnia listy użytkowników na które zalogowanie się powiodło użyłem polecenia:
\begin{minted}{awk}
    grep 'Accepted' secure | awk '{print $9}' | sort | uniq -c
    grep 'Accepted' secure.1 | awk '{print $9}' | sort | uniq -c
\end{minted}

\subsubsection{Wyniki}
W przypadku obu plików wyniki są podobne w pliku secure udało się zalogować na konto root 1 raz, natomiast w pliku secure.1 udało się zalogować na konto local\_user 29 razy

\subsubsection{Wnioski}

Ataki słownikowe nie powiodły się ponieważ wykorzystywane jest logowanie poprzez klucz prywatny-publiczny.
\section{Zadanie 2}
Zadanie drugie polegało na przeprowadzeniu analizy logów serwisu SMTP. Zadanie zawierało plik final.log
\subsection{Podpunkt 1}
W celu wyświelnia liczby niudanych prób logowania do serwisu SMTP użyłem polecenia:
\begin{minted}{awk}
    grep "login authenticator failed" final.log -c
\end{minted}
\subsubsection{Wyniki}
Wynik jaki otrzymałem to  234
\subsection{Podpunkt 2}
W celu wyświetlenia listy adresów ip z których nastąpiły błędne próby logowania użyłem polecenia:
\begin{minted}{awk}
    grep "login authenticator failed" final.log | grep -E '[0-9]{1,3}\.[0-9]{1,3}\.[0-9]{1,3}\.[0-9]{1,3}' -o | sort | uniq -c | sort -nr
\end{minted}
\subsubsection{Wyniki}
\begin{minted}{awk}
    69 127.0.0.1
    30 142.11.199.241
    27 185.180.222.147
    20 185.222.209.202
    18 10.0.0.142
    12 185.222.209.78
    10 94.102.49.198
     9 128.106.1.6
     8 91.212.150.81
     8 185.222.209.201
     7 93.157.63.30
     7 64.235.38.22
     5 185.211.245.195
     4 93.157.63.9
     4 93.157.63.8
     4 93.157.63.7
     4 93.157.63.6
     4 92.246.76.92
     4 80.85.153.204
     4 80.82.65.187
     4 62.50.131.54
     4 185.231.245.46
     4 185.231.245.44
     4 185.231.245.43
     4 185.231.245.41
     4 185.144.29.219
     4 185.144.28.111
     3 80.85.153.211
     3 80.85.153.206
     3 193.233.74.17
     3 185.231.245.49
     3 185.231.245.48
     3 185.231.245.40
     3 185.144.30.39
     3 185.144.29.30
     3 185.144.29.178
     2 37.120.146.84
     2 185.231.245.50
     2 185.144.28.241
     1 92.61.148.10
     [...]
     1 103.57.195.147
\end{minted}
\subsection{Podpunkt 3}
W celu wyświetlenia listy krajów z których nastąpiły błędne próby logowania użyłem polecenia:
\begin{minted}{awk}
    grep "login authenticator failed" final.log | grep -Eo '[0-9]{1,3}\.[0-9]{1,3}\.[0-9]{1,3}\.[0-9]{1,3}' | xargs -I{} geoiplookup {} | sort | uniq -c | sort -nr | head -n 10
\end{minted}
\subsubsection{Wyniki}
\begin{minted}{awk}
    105 GeoIP Country Edition: RU, Russian Federation
    88 GeoIP Country Edition: IP Address not found
    46 GeoIP Country Edition: NL, Netherlands
    44 GeoIP Country Edition: GB, United Kingdom
    38 GeoIP Country Edition: US, United States
    11 GeoIP Country Edition: VN, Vietnam
     9 GeoIP Country Edition: SG, Singapore
     4 GeoIP Country Edition: EG, Egypt
     2 GeoIP Country Edition: TH, Thailand
     2 GeoIP Country Edition: DE, Germany
\end{minted}
Jak można zauważyć próby ataku przedewszystkim pochodzą z Rosji, Holandii, Wielkiej Brytanii, Stanów Zjednoczonych.
\subsection{Podpunkt 4}
W celu wyświetlenia listy lokalnych użytkowników użyłem polecenia:
\begin{minted}{awk}
    grep "R=localuser" final.log | cut -d ' ' -f 5 | cut -d "@" -f 1 | sort | uniq | sort -t "-" -k 2 -n
\end{minted}
\subsubsection{Wyniki}
Konta lokalne to user-3, user-4, user-5, user-6, user-7, user-8, user-9, user-10, user-11, user-12, user-13, user-14, user-16, user-17, user-18, user-19, user-20, user-23, user-24, user-25, user-25-jg, user-25-lg, user-25-wb, user-30, user-32, user-34, user-35, user-40, user-41, user-43, user-44, user-45, user-47, user-48, user-50, user-51, user-53, user-54, user-55, user-56, user-57, user-58, user-59, user-60, user-61, user-68, user-69, user-70, user-71, user-72, user-73, user-74, user-75, user-76, user-77, user-78, user-79, user-81, user-82, user-83, user-84, user-86, user-87, user-88, user-89, user-90, user-91, user-93, user-94, user-95, user-96, user-97, user-99
\subsection{Podpunkt 5}
W celu wyświetlenia listy użytkowników lokalych na których było najwięcej prób ataku użyłem polecenia:
\begin{minted}{awk}
    grep "login authenticator failed" final.log | grep -v "@" | awk '{print $NF}' | cut -d '=' -f 2 | cut -d ')' -f 1 | sort | uniq -c | sort -nr | head -n 20
\end{minted}
\subsubsection{Wyniki}
\begin{minted}{awk}
    72 user-81
     11 user-25-jg
     10 user-25
      5 vermont
      4 arthur
      2 user-54
      2 user-5
      2 user-25-wb
      2 user-25-lg
      2 user-10
      1 yangxiong
      1 via.postal
      1 user-9
      1 user-76
      1 user-74
      1 user-72
      1 user-71
      1 user-56
      1 toiawase
      1 tim_osborn1
\end{minted}
\subsubsection{Wnioski}
Jak można zauważyć najwięcej prób ataku były przeprowadzane na konto `user-81", prawopodobnie zostało skompromitowane.
\subsection{Podpunkt 6}
W celu wyświetlenia prób wykorzystywania serwera jako `open relay" użyłem kilku poleceń:
\begin{minted}{awk}
    grep "no host name found for IP address" final.log
    grep "rejected RCPT" final.log
    grep "rejected unknown sender" final.log
    grep "message too big for system" final.log
    grep "max connection rate exceeded" final.log
    grep "authentication failed" final.log
\end{minted}
Jak można zauważyć użyłem 6 komend które pozwolą nam na sprawdzenie czy serwer jest wykorzystywany jako "open relay", jednakże nie jest to jednoznaczne.
Komenda 1 pozwala sprawdzić czy serwer nie jest w stanie zidentyfikować adresu IP.
Komenda 2 pozwala sprawdzić czy serwer odrzuca wiadomości.
Komenda 3 pozwala sprawdzić czy serwer odrzuca nieznanych nadawców.
Komenda 4 pozwala sprawdzić czy serwer odrzuca wiadomości które są za duże.
Komenda 5 pozwala sprawdzić czy serwer odrzuca wiadomości które są wysyłane zbyt szybko.
Komenda 6 pozwala sprawdzić czy serwer odrzuca próby logowania.
\subsubsection{Wyniki}
Nie dla wszystkich komend udało mi się uzyskać wyniki, jednakże dla komend w 1 i drugiej linijce udało mi się uzyskać wyniki.
\newline
Wyniki dla komendy `grep "no host name found for IP address" final.log`:
\begin{minted}{awk}
    2019-03-08 04:06:06 no host name found for IP address 185.222.209.78
    2019-03-08 04:06:08 no host name found for IP address 185.222.209.78
    2019-03-08 04:06:40 no host name found for IP address 185.222.209.78
    2019-03-08 04:06:42 no host name found for IP address 185.222.209.78
    2019-03-08 04:08:24 no host name found for IP address 185.222.209.78
    [...]
    2019-03-18 18:32:08 no host name found for IP address 10.0.0.92
    2019-03-18 18:37:42 no host name found for IP address 10.0.0.92
    2019-03-18 21:20:01 no host name found for IP address 185.222.209.202
    2019-03-19 00:47:53 no host name found for IP address 185.231.245.41
    2019-03-19 00:47:53 no host name found for IP address 185.231.245.41
    2019-03-19 00:47:53 no host name found for IP address 185.231.245.41
    2019-03-19 02:55:01 no host name found for IP address 91.212.150.81
\end{minted}
Wyniki dla komendy `grep "rejected RCPT" final.log`:
\begin{minted}{awk}
    2019-03-08 04:42:04 H=smtp.dzim.zarow.pl [51.38.142.82] F=<foreign-user-2641@sermn.zagan.pl> temporarily rejected RCPT <user-25-wb@some-domain.pl>: Could not complete sender verify
    2019-03-08 05:42:25 H=(maerke.nl) [31.132.218.252] F=<foreign-user-1378@maerke.nl> rejected RCPT <user-25-lg@some-domain.pl>: Access denied - 31.132.218.252 listed by dnsbl.sorbs.net
    2019-03-08 05:42:25 H=(maerke.nl) [31.132.218.252] F=<foreign-user-1378@maerke.nl> rejected RCPT <user-25-jg@some-domain.pl>: Access denied - 31.132.218.252 listed by dnsbl.sorbs.net
    2019-03-08 05:42:25 H=(maerke.nl) [31.132.218.252] F=<foreign-user-1378@maerke.nl> rejected RCPT <user-25@some-domain.pl>: Access denied - 31.132.218.252 listed by dnsbl.sorbs.net
    2019-03-08 05:42:25 H=(maerke.nl) [31.132.218.252] F=<foreign-user-1378@maerke.nl> rejected RCPT <user-25-wb@some-domain.pl>: Access denied - 31.132.218.252 listed by dnsbl.sorbs.net
    2019-03-08 05:52:03 H=smtp.dzim.zarow.pl [51.38.142.82] F=<foreign-user-2641@sermn.zagan.pl> temporarily rejected RCPT <user-25-wb@some-domain.pl>: Could not complete sender verify
    2019-03-18 18:02:31 H=(125-209-69-138.multi.net.pk) [202.142.163.62] F=<foreign-user-1395@multi.net.pk> rejected RCPT <user-25-wb@some-domain.pl>: Access denied - 202.142.163.62 listed by dnsbl.sorbs.net
    [...]
    2019-03-19 02:34:44 H=rrcs-162-155-179-211.cenhidden_useral.biz.rr.com [162.155.179.211] F=<foreign-user-1199@rr.com> rejected RCPT <user-25@some-domain.pl>: Access denied - 162.155.179.211 listed by b.barracudacenhidden_useral.org
    2019-03-19 02:34:45 H=rrcs-162-155-179-211.cenhidden_useral.biz.rr.com [162.155.179.211] F=<foreign-user-1199@rr.com> rejected RCPT <user-25-wb@some-domain.pl>: Access denied - 162.155.179.211 listed by b.barracudacenhidden_useral.org
    2019-03-19 02:43:33 H=(168-195-135210.deltanetworks.net.br) [168.195.135.210] F=<foreign-user-1360@deltanetworks.net.br> rejected RCPT <user-25-lg@some-domain.pl>: Sender verify failed
    2019-03-19 02:43:33 H=(168-195-135210.deltanetworks.net.br) [168.195.135.210] F=<foreign-user-1360@deltanetworks.net.br> rejected RCPT <user-25-jg@some-domain.pl>: Sender verify failed
    2019-03-19 02:43:34 H=(168-195-135210.deltanetworks.net.br) [168.195.135.210] F=<foreign-user-1360@deltanetworks.net.br> rejected RCPT <user-25@some-domain.pl>: Sender verify failed
    2019-03-19 02:43:34 H=(168-195-135210.deltanetworks.net.br) [168.195.135.210] F=<foreign-user-1360@deltanetworks.net.br> rejected RCPT <user-25-wb@some-domain.pl>: Sender verify failed
    2019-03-19 03:43:42 H=infinity16p.cf16.pl [94.177.240.244] F=<foreign-user-1768@financial-hidden_users.pl> rejected RCPT <user-25-lg@some-domain.pl>: Access denied - 94.177.240.244 listed by dnsbl.sorbs.net
\end{minted}
\subsubsection{Wnioski}
Jak można zauważyć w obu przypadkach logi są długie(800+ lini) co pokazuje nam że serwer prawopodobnie 
jest wykorzystywany jako "open relay". 
Jednakże aby mieć pewność należałoby przeprowadzić analizę logów sieciowych na portach 25 i 587.

\end{document}
